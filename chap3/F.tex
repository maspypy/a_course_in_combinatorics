\subsection{}
(1)2色をa,bとする。1がaのとき、1+1=2より2はb、2+2=4より4はa、1+3=4より3はbとなるが、1+4=2+3=5より、5をa,bどちらで塗っても単色等式が成立する。

(2)3色をa,b,cとする。全探索をするプログラムを書くと、1から順に\\
\{a,b,b,a,c,c,a,c,c,a,b,b,a\}と彩色することにより、13以下の数からなる単色等式が存在しないようにできることが確かめられる。
