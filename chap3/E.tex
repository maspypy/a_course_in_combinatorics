\subsection{}
\begin{enumerate}
 \item 不等式は $2^k\leq n$ と書き換えられる.
 
 $k$ についての帰納法で示す.$k=0$ はよい.$k\geq 1$ として $k-1$ での成立を仮定.$v_0\in K_n$ を任意にとる.
 \begin{align*}
  V_1 = \{w\mid \text{$v$ から $w$ へ向き付けられている}\},\\
  V_2 = \{w\mid \text{$w$ から $v$ へ向き付けられている}\}
 \end{align*}
 とすると,$|V_1| + |V_2| = n-1 \geq 2^{k}-1$ であるから,$V_1$ または $V_2$ のどちらかは $2^{k-1}$ 個以上の頂点を含む.
 その部分集合に帰納法の仮定を用いて $k-1$ 個の頂点からなる推移的トーナメントをとり,$v$ を合わせれば $k$ 個の頂点からなる推移的トーナメントが得られる.
 \item 不等式は $n^k < 2^{\comb{k}{2}}$ と言い換えられる.$n^k \geq \comb{n}{k}\cdot k!$ であるから,$\comb{n}{k}\cdot \frac{k!}{2^{\comb{k}{2}}} < 1$ が成り立つ.
 
 $K_n$ の各辺を,乱択で向き付けることを考える.$k$ 個の頂点集合に対して,それが推移的トーナメントになる確率は,$\frac{k!}{2^{\comb{k}{2}}}$ である(強い順に並べる方法が $k!$ 通りあり,
 向き付けの方法が $2^{\comb{k}{2}}$ 通りある).したがって,$\comb{n}{k}\cdot \frac{k!}{2^{\comb{k}{2}}}$ は,$k$ 個の頂点からなる推移的トーナメントの個数の期待値に等しい.
 この値が $1$ 未満であることから,ある向き付けに対して $k$ 個の頂点からなる推移的トーナメントの非存在が従う.
\end{enumerate}