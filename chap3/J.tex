\subsection{}
このままだと $n=1$ で自明に嘘だったので, $k> \sqrt{2n} - 1$ で示します.

赤三角形のない極大誘導部分グラフを $H$ とする(空集合を頂点集合とする誘導部分グラフが条件を満たすので,極大なものは存在する).
任意の $v\in G-H$ に対して,$H+v$ が赤三角形を含む.したがって,$e=ab\in H$ であって,$ab, av, bv$ がすべて赤いものが存在する.
このような $e$ をひとつとり $f(v)$ とすることで,写像 $f\colon V(G-H) \longrightarrow E(H)$ を定める.

問題の条件(同一の辺が $2$ 個の赤三角形に含まれない)から,$f$ は単射である.したがって $|V(G-H)| \leq |E(H)|$.
$|V(H)| = k$ とすると,$n - k\leq \binom{k}{2}$ となる.
$2n \leq k(k+1)$ より $\sqrt{2n} < k+1$ となり $k > \sqrt{2n} - 1$ が分かる.
