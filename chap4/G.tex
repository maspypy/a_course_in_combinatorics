\subsection{}
Mantel の証明のほぼ単純な拡張である.
各頂点 $v$ に非負の重み $z_v$ (全頂点の重みの和は $1$)を与えて,$\sum_{e=ab}z_az_b$ の最大値が達成されている状況を考える.
そのうち,正の重みの頂点数が最小のものをとると,正の重みの頂点数が $p-1$ 以下であることが分かる
($p$ 個以上あれば,クリークの非存在より非隣接点に正の重みが乗っていて,それをまとめて頂点数が減らせる).

$k = p-1$ とし,それらの重みを $z_1,\ldots,z_k$ とする.
$\sum_{i\neq j}z_iz_j = \frac12((z_1+\cdots+z_k)^2 - (z_1^2+\cdots+z_k)^2)$ であるが,コーシー・シュワルツの不等式などから
$z_1^2+\cdots+z_k^2 \geq \frac{1}{k}(z_1+\cdots+z_k)^2$ が分かるので,$z_1+\cdots+z_k=1$ と合わせて $\sum_{i\neq j}z_iz_j \leq \frac12(1-\frac{1}{k})$ である.

最大値を達成する割り当てでこの不等式が成り立つので,任意の割り当てでも成り立つ.特に全頂点に重み $\frac{1}{n}$ を割り当てたときのことを考えて
$\frac{|E(G)|}{n^2}\leq \frac{1}{2}(1-\frac{1}{k})$ が成り立つ.これに $k=p-1$ を代入して整理すればよい.