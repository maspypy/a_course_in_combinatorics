\subsection{}
$2n$ 頂点の $(n-1)$-正則二部グラフ $G_n$ をとる.(1), (2) ではこれの $n=4, 5$ の場合が問題となっている.

$G_n$ は完全二部グラフ $K_{n,n}$ の部分グラフと見なせる.よって $K_{n,n}$ から完全マッチングをひとつ除いたものと見なせる.
よって頂点の名前を上手く付けることで,
\begin{itemize}
 \item $G_n$ は部集合 $A = \{v_1,v_2,\ldots,v_n\}$, $B = \{w_1,w_2,\ldots,w_n\}$ を持つ二部グラフ.
 \item $v_iw_j\in G_n \iff i \neq j$.
\end{itemize}
が成り立つ(特に $G_n$ は同型を除き一意に定まる).

完全マッチングは,$\{v_iw_j\}$ に対して写像 $f\colon i\mapsto j$ を対応させることで,完全マッチングは
不動点を持たない全単射 $f\colon \{1,2,\ldots,n\}\longrightarrow  \{1,2,\ldots,n\}$ と一対一に対応する.

この数え上げは,完全順列(攪乱順列)の数え上げとして有名である.
答は $\sum_{k=0}^n (-1)^k\frac{n!}{k!}$ であり,(1), (2) の答は $n=4, 5$ の場合 (1)9, (2)44 である.